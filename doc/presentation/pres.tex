% Created 2015-03-13 Fri 15:17
\documentclass[xcolor=dvipsnames]{beamer}
\usepackage[utf8]{inputenc}
\usepackage[T1]{fontenc}
\usepackage{fixltx2e}
\usepackage{graphicx}
\usepackage{longtable}
\usepackage{float}
\usepackage{wrapfig}
\usepackage{rotating}
\usepackage[normalem]{ulem}
\usepackage{amsmath}
\usepackage{textcomp}
\usepackage{marvosym}
\usepackage{wasysym}
\usepackage{amssymb}
\usepackage{hyperref}
\tolerance=1000
\mode<beamer>{\usetheme{boxes}}
\setbeamertemplate{items}[circle]
\setbeamertemplate{navigation symbols}{}
\usecolortheme[named=RoyalBlue]{structure}
\setbeamercolor{boxcol}{fg=black,bg=black!10}
\setbeamercolor{lowercolgreen}{fg=black,bg=green!10}
\usepackage{pslatex}
\usepackage{amsfonts}
\usepackage{booktabs}
\usepackage{scripttab}
\usepackage{tabularx}
\usepackage[T1]{fontenc}
\usepackage{lmodern}
\renewcommand{\vector}     {\boldsymbol}
\newcommand{\uvec}            {\vector{u}}
\newcommand{\vvec}            {\vector{v}}
\newcommand{\wvec}            {\vector{w}}
\newcommand{\svec}            {\vector{s}}
\newcommand{\Bili}[2]      {\left(#1,#2\right)}
\institute{ AM574 Conservation Laws and Finite Volume Methods \\
University of Washington, Seattle USA}
\input{../macros/RJLmacros}
\usetheme{default}
\author{Xinsheng Qin, Kaspar Müller}
\date{March 13, 2015}
\title{Hydraulic bore interaction with a column \\ A comparison between the solution of the shallow equation and experimental results}
\hypersetup{
  pdfkeywords={},
  pdfsubject={},
  pdfcreator={Emacs 23.4.1 (Org mode 8.2.5g)}}
\begin{document}

\maketitle
\begin{frame}{Outline}
\tableofcontents
\end{frame}

\usebackgroundtemplate{\includegraphics[width=\paperwidth]{figures/back02.pdf}}

\section{}
\label{sec-1}
\section{Introduction}
\label{sec-2}
\begin{frame}[label=sec-2-1]{Introduction}
Comparison between Solution of the shallow water equations solved with CLAWPACK and experimental results.
\begin{itemize}
\item Experiment by Halldór Árnason
\item Comaprison of water level at various locations
\item 3 Cases
\begin{enumerate}
\item Dam Break
\item Dam Break with square column
\item Dam Break with cylindrical column
\end{enumerate}
\end{itemize}
\end{frame}

\section{The Model}
\label{sec-3}
\begin{frame}[label=sec-3-1]{The Model equations}
\begin{block}{2D depth averaged  shallow water equations}
\begin{eqnarray*}
h_t + (uh)_x + (vh)_y &=& 0\nono\\
(hu)_t + (huv)_y +(hu^2 + \frac{1}{2}gh^2)_x &=& -ghB_x - Du\\
(hv)_t + (huv)_x +(hv^2 + \frac{1}{2}gh^2)_y &=& -ghB_y - Dv
\end{eqnarray*}
where $B$ is the topography and $D$ the drag coefficient. $g$ stands for the gravitational acceleration.
\end{block}
\end{frame}

\section{Framework and Method}
\label{sec-4}
\begin{frame}[label=sec-4-1]{CLAWPACK/GeoClaw}
\begin{block}{GeoClaw}
\begin{itemize}
\item Geophysical flow problems
\item Specialized version of CLAPACK and AMRClaw
\end{itemize}
\end{block}

\begin{block}{Requirements for solver}
\begin{itemize}
\item Flow over topography
\item Handle non-trivial steady state
\begin{itemize}
\item Ocean at rest
\end{itemize}
\item Dry state handling
\item Multiple scales in space and time
\end{itemize}
\end{block}
\end{frame}

\begin{frame}[label=sec-4-2]{Riemann solver}
\begin{block}{Approaches}
\begin{itemize}
\item f-wave approach
\begin{itemize}
\item guarantees numerical conservation
\item source terms simply included in solver
\item not clear how to prevent depth negativity
\end{itemize}
\item HLLE solver
\begin{itemize}
\item guarantees depth non-negativity
\item fails to capture large transonic rarefactions
\item not well balanced for steady states
\end{itemize}
\end{itemize}
\end{block}

\begin{block}{Augmented Riemann solver}
\begin{itemize}
\item Split into more than two waves
\begin{itemize}
\item Add 1 wave to have two waves to represent large transonic rarefactions
\item Add 1 wave to incorporate topography - balance for steady state
\end{itemize}
\end{itemize}
\end{block}
\end{frame}

\begin{frame}[label=sec-4-3]{Wave propagation algorithm}
\begin{itemize}
\item 2D hyperbolic system
\end{itemize}
\begin{equation*}
q_t + A(q,x,y)q_x + B(q,x,y)q_y = 0
\end{equation*}
\begin{itemize}
\item Approximation of the state
\end{itemize}
\begin{eqnarray*}
\qijnp = \qijn &-& \dtdx \left(\apdqimj^n + \amdqiphj^n\right) \\
               &-& \dtdy \left(\bpdqijm^n + \bmdqijph^n\right)\\
               &-&\dtdx \left(\tFiphj^n - \tFimhj^n\right) - \dtdy \left(\tGijph^n - \tGijmh^n\right)
\end{eqnarray*}
\begin{itemize}
\item Fluctuations ${\cal A}^\pm\Delta \Q_{i\mp 1/2,j}^n$ and ${\cal B}^\pm\Delta \Q_{i,j\mp 1/2,j}^n$
\item Second order correction terms $\tilde F_{i\pm 1/2,j}^n$ and $\tilde G_{i,j\pm 1/2}^n$
\end{itemize}
\end{frame}

\begin{frame}[label=sec-4-4]{Approach in 1D}
\begin{block}{f-wave approach}
\begin{equation*}
f(\qi) -f(\qim) = \psum \WZ_{i-1/2}^p = \psum \beta^p_{i-1/2}r^p_{i-1/2}
\end{equation*}
\begin{itemize}
\item jumps in the flux instead of state
\end{itemize}
\end{block}

\begin{columns}
\begin{column}{0.4\textwidth}
\begin{block}{Fluctuations}
\begin{equation*}
\begin{align}
\amdqimh &= \sum_{p:s^p_{i-1/2}<0} \WZ_{i-1/2}^p \\
\apdqimh &= \sum_{p:s^p_{i-1/2}>0} \WZ_{i-1/2}^p  
\end{align}
\end{equation*}
\end{block}
\end{column}

\begin{column}{0.5\textwidth}
\begin{block}{Decomposition into waves}
\begin{equation*}
\begin{bmatrix}
H_i - H_{i-1}\\
HU_i - HU_{i-1}\\
\varphi(\qi) - \varphi(\qim) \\
B_i - B_{i-1}\\
\end{bmatrix}
= \sum_{p=0}^3 \alpha^p_{i-1/2}w^p_{i-1/2}
\end{equation*}
\end{block}
\end{column}
\end{columns}
\end{frame}

\section{Test cases}
\label{sec-5}
\begin{frame}[label=sec-5-1]{Setup of the test case}
\begin{columns}
\begin{column}{1.0\textwidth}
\begin{block}{}
\includegraphics[width=.9\linewidth]{../proposal/figures/Diagram_case.jpg} \\
\end{block}
\end{column}
\end{columns}
\end{frame}

\begin{frame}[label=sec-5-2]{Case01 - Dam Break}
\begin{columns}
\begin{column}{0.5\textwidth}
\begin{block}{\qquad\qquad 0 $\le$ t $\le$ 18}
\includegraphics[width=.9\linewidth]{../paper/plots/waveheight_nocolumn.png} \\
\begin{itemize}
\item Wave arrives at t=3.8s
\item Second jump from reflection from right wall
\end{itemize}
\end{block}
\end{column}

\begin{column}{0.6\textwidth}
\begin{block}{\qquad\qquad 3 $\le$ t $\le$ 4}
\includegraphics[width=.9\linewidth]{../paper/plots/waveheight_nocolumn_zoomin.png}  \\
\begin{itemize}
\item GeoClaw 0.02s ahead
\item Peak value overestimated
\end{itemize}
\vskip5mm
\end{block}
\end{column}
\end{columns}

\begin{block}{}
\href{../paper/plots/animation/cross_section_nocolumn.mp4}{movie}
\end{block}
\end{frame}

\begin{frame}[label=sec-5-3]{Case02 - Dam Break with Square Column}
\begin{columns}
\begin{column}{0.5\textwidth}
\begin{block}{}
\includegraphics[width=.9\linewidth]{../paper/plots/surface_elevation_square_nearcolumn/frame0009fig1.png}\\
    \includegraphics[width=.9\linewidth]{../paper/plots/surface_elevation_square_nearcolumn/frame0010fig1.png}\\
\end{block}
\end{column}

\begin{column}{0.5\textwidth}
\begin{block}{}
\includegraphics[width=.9\linewidth]{../paper/plots/surface_elevation_square_nearcolumn/frame0011fig1.png}\\
    \includegraphics[width=.9\linewidth]{../paper/plots/surface_elevation_square_nearcolumn/frame0012fig1.png}\\
\end{block}
\end{column}
\end{columns}

\begin{block}{}
\href{../paper/plots/animation/surface_elevation_square_nearcolumn.mp4}{movie}
\end{block}
\end{frame}

\begin{frame}[label=sec-5-4]{Case02 - Dam Break with Square Column}
\begin{columns}
\begin{column}{0.5\textwidth}
\begin{block}{}
\includegraphics[width=.9\linewidth]{../paper/plots/waveheight_square_x11.png} \\
\end{block}
\end{column}

\begin{column}{0.5\textwidth}
\begin{block}{}
\includegraphics[width=.9\linewidth]{../paper/plots/waveheight_square_x1102.png}  \\
\end{block}
\end{column}
\end{columns}

\begin{block}{}
\begin{itemize}
\item Two jumps in GeoClaw result
\begin{itemize}
\item One originates in dam break
\item Another results from wave reflected by column
\end{itemize}
\item Time interval becomes shorter
\item Difference between GeoClaw result and experiment result:
\begin{itemize}
\item Shock shape
\item Arrival time of the shock
\end{itemize}
\end{itemize}
\end{block}
\end{frame}

\begin{frame}[label=sec-5-5]{Case02 - Dam Break with Square Column}
\begin{columns}
\begin{column}{0.5\textwidth}
\begin{block}{}
\includegraphics[width=.9\linewidth]{../paper/plots/waveheight_square_x1104_largerTimeRange.png} \\
\begin{itemize}
\item Wave height at front face
\item Wave height at back face
\item Comparison between cases with and without column
\end{itemize}
\vskip13mm
\end{block}
\end{column}

\begin{column}{0.5\textwidth}
\begin{block}{}
\includegraphics[width=.9\linewidth]{../paper/plots/waveheight_square_back.png}  \\
     \includegraphics[width=.9\linewidth]{../paper/plots/waveheight_x11_comparison.png}  \\
\end{block}
\end{column}
\end{columns}
\end{frame}
\begin{frame}[label=sec-5-6]{Case03 - Dam Break with Cylindrical Column}
\begin{columns}
\begin{column}{0.5\textwidth}
\begin{block}{}
\includegraphics[width=.9\linewidth]{../paper/plots/surface_elevation_cylinder_nearcolumn/frame0014fig1.png}\\
    \includegraphics[width=.9\linewidth]{../paper/plots/surface_elevation_cylinder_nearcolumn/frame0015fig1.png}\\
\end{block}
\end{column}

\begin{column}{0.5\textwidth}
\begin{block}{}
\includegraphics[width=.9\linewidth]{../paper/plots/surface_elevation_cylinder_nearcolumn/frame0016fig1.png}\\
    \includegraphics[width=.9\linewidth]{../paper/plots/surface_elevation_cylinder_nearcolumn/frame0017fig1.png}\\
\end{block}
\end{column}
\end{columns}

\begin{block}{}
\href{../paper/plots/animation/surface_elevation_cylinder_nearcolumn.mp4}{movie}
\end{block}
\end{frame}

\begin{frame}[label=sec-5-7]{Case03 - Dam Break with Cylindrical Column}
\begin{columns}
\begin{column}{0.5\textwidth}
\begin{block}{}
\includegraphics[width=.9\linewidth]{../paper/plots/waveheight_cylinder_x11_finer_new.png} \\
\begin{itemize}
\item Mesh refinement
\item History of wave height at front face and back of the column
\end{itemize}
\vskip15mm
\end{block}
\end{column}

\begin{column}{0.5\textwidth}
\begin{block}{}
\includegraphics[width=.9\linewidth]{../paper/plots/waveheight_cylinder_x1104_largerTimeScale_finer.png}  \\
     \includegraphics[width=.9\linewidth]{../paper/plots/waveheight_cylinder_x1116_finer.png} \\
\end{block}
\end{column}
\end{columns}

\begin{block}{}
\vskip5mm
\end{block}
\end{frame}
\section{Outlook}
\label{sec-6}
\begin{frame}[label=sec-6-1]{Conclusion and Outlook}
\begin{itemize}
\item For the dam break problem, GeoClaw underestimates the arrival time of the shock by only 0.02s
\item In both cases with cylinder and square column, GeoClaw overestimates the arrival time of the shock
\item In general GeoClaw agrees well with the measured water levels at various locations for the square as well as the cylindrical column
\end{itemize}
\begin{block}{Future work}
\begin{itemize}
\item A mapped grid could be used to improve the solution in cylindrical column case
\item Comparing the forces on the structure by using $F =\frac{1}{2}C_D\rho d hu^2$
\end{itemize}
\end{block}
\end{frame}
% Emacs 23.4.1 (Org mode 8.2.5g)
\end{document}