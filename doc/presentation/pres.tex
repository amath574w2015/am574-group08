% Created 2015-03-12 Thu 16:59
\documentclass[xcolor=dvipsnames]{beamer}
\usepackage[utf8]{inputenc}
\usepackage[T1]{fontenc}
\usepackage{fixltx2e}
\usepackage{graphicx}
\usepackage{longtable}
\usepackage{float}
\usepackage{wrapfig}
\usepackage{rotating}
\usepackage[normalem]{ulem}
\usepackage{amsmath}
\usepackage{textcomp}
\usepackage{marvosym}
\usepackage{wasysym}
\usepackage{amssymb}
\usepackage{hyperref}
\tolerance=1000
\mode<beamer>{\usetheme{boxes}}
\setbeamertemplate{items}[circle]
\setbeamertemplate{navigation symbols}{}
\usecolortheme[named=RoyalBlue]{structure}
\setbeamercolor{boxcol}{fg=black,bg=black!10}
\setbeamercolor{lowercolgreen}{fg=black,bg=green!10}
\usepackage{pslatex}
\usepackage{amsfonts}
\usepackage{booktabs}
\usepackage{scripttab}
\usepackage{tabularx}
\usepackage[T1]{fontenc}
\usepackage{lmodern}
\renewcommand{\vector}     {\boldsymbol}
\newcommand{\uvec}            {\vector{u}}
\newcommand{\vvec}            {\vector{v}}
\newcommand{\wvec}            {\vector{w}}
\newcommand{\svec}            {\vector{s}}
\newcommand{\Bili}[2]      {\left(#1,#2\right)}
\institute{ AM574 Conservation Laws and Finite Volume Methods \\
University of Washington, Seattle USA}
\input{../macros/RJLmacros}
\usetheme{default}
\author{Xinsheng Qin, Kaspar Müller}
\date{March 13, 2015}
\title{Hydraulic bore interaction with a column - A comparison between the solution of the shallow equation and experimental results}
\hypersetup{
  pdfkeywords={},
  pdfsubject={},
  pdfcreator={Emacs 23.4.1 (Org mode 8.2.5g)}}
\begin{document}

\maketitle
\begin{frame}{Outline}
\tableofcontents
\end{frame}

\usebackgroundtemplate{\includegraphics[width=\paperwidth]{figures/back02.pdf}}

\section{}
\label{sec-1}
\section{Introduction}
\label{sec-2}
\begin{frame}[label=sec-2-1]{Introduction}
\end{frame}

\section{The Model}
\label{sec-3}
\begin{frame}[label=sec-3-1]{The Shallow Water Equations in 2D}
\begin{eqnarray*}
h_t + (uh)_x + (vh)_y &=& 0\nono\\
(hu)_t + (huv)_y +(hu^2 + \frac{1}{2}gh^2)_x &=& -ghB_x - Du\\
(hv)_t + (huv)_x +(hv^2 + \frac{1}{2}gh^2)_y &=& -ghB_y - Dv
\end{eqnarray*}
where $B$ is the topography and $D$ the drag coefficient. $g$ stands for the gravitational acceleration.
\end{frame}

\section{Framework and Method}
\label{sec-4}
\begin{frame}[label=sec-4-1]{CLAWPACK/GeoClaw}
\begin{block}{Software framework}
\end{block}
\begin{block}{Method}
\end{block}
\end{frame}

\section{Test cases}
\label{sec-5}
\begin{frame}[label=sec-5-1]{Setup of the test case}
\begin{columns}
\begin{column}{1.0\textwidth}
\begin{block}{}
\includegraphics[width=.9\linewidth]{../proposal/figures/Diagram_case.jpg} \\
\end{block}
\end{column}
\end{columns}
\end{frame}

\begin{frame}[label=sec-5-2]{Case01 - Dam Break}
\begin{columns}
\begin{column}{0.5\textwidth}
\begin{block}{\qquad\qquad 0 $\le$ t $\le$ 18}
\includegraphics[width=.9\linewidth]{../paper/plots/waveheight_nocolumn.png} \\
\begin{itemize}
\item Wave arrives at t=3.8s
\item Second jump from reflection from right wall
\end{itemize}
\end{block}
\end{column}

\begin{column}{0.5\textwidth}
\begin{block}{\qquad\qquad 3 $\le$ t $\le$ 4}
\includegraphics[width=.9\linewidth]{../paper/plots/waveheight_nocolumn_zoomin.png}  \\
\begin{itemize}
\item Geoclaw 0.02s ahead
\item Peak value overestimated
\end{itemize}
\vskip5mm
\end{block}
\end{column}
\end{columns}
\end{frame}
\begin{frame}[label=sec-5-3]{Case02 - Dam Break with Square Column}
\begin{columns}
\begin{column}{0.5\textwidth}
\begin{block}{}
\includegraphics[width=.9\linewidth]{../paper/plots/waveheight_square_x11.png} \\
\end{block}
\end{column}

\begin{column}{0.5\textwidth}
\begin{block}{}
\includegraphics[width=.9\linewidth]{../paper/plots/waveheight_square_x1102.png}  \\
\end{block}
\end{column}
\end{columns}

\begin{block}{}
\begin{itemize}
\item Second jump reflection from the column
\item Smooth increase in wave height in experiment
\end{itemize}
\end{block}
\end{frame}

\begin{frame}[label=sec-5-4]{Case03 - Dam Break with Cylindrical Column}
\begin{columns}
\begin{column}{0.5\textwidth}
\begin{block}{}
\includegraphics[width=.9\linewidth]{../paper/plots/waveheight_cylinder_x11.png} \\
\end{block}
\end{column}

\begin{column}{0.5\textwidth}
\begin{block}{}
\includegraphics[width=.9\linewidth]{../paper/plots/waveheight_cylinder_x1104_largerTimeScale.png}  \\
\vskip5mm
\end{block}
\end{column}
\end{columns}
\end{frame}
\section{Outlook}
\label{sec-6}
\begin{frame}[label=sec-6-1]{Conclusion and Outlook}
\end{frame}
% Emacs 23.4.1 (Org mode 8.2.5g)
\end{document}