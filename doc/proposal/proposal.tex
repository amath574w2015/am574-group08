\documentclass[11pt]{article}

\usepackage{graphicx}
\usepackage{amsmath,amsfonts,amssymb}

\usepackage{hyperref}  % for urls and hyperlinks


\setlength{\textwidth}{6.2in}
\setlength{\oddsidemargin}{0.3in}
\setlength{\evensidemargin}{0in}
\setlength{\textheight}{8.7in}
\setlength{\voffset}{-.7in}
\setlength{\headsep}{26pt}
\setlength{\parindent}{0pt}
\setlength{\parskip}{5pt}

% input some useful macros from RJLmacros.tex:
\input{../macros/RJLmacros}

\begin{document}
\section{Introduction}
We are planing to model the interaction of a bore and a free-standing coastal structure with CLAWPACK in 2D.
Simulation work can be compared to benchmark experiment by Halld\'or \'Arnason \cite{HA}.
\section{Description}
The benchmark experiment was conducted in a 16.6 m long, 0.6m wide and 0.45 m deep wave tank. When a gate is lifted, it generates a single bore to impinge the column located downstream.
We are going to simulate 2 cases in the experiment. The 1st case purely generated the single bore, the 2nd case generated the bore to impinge the column with one side facing the flow.
\par
In the 1st case, wave height and streamwise velocity, U, at different depth, at where the center of column should be located are measured. In the 2nd case, the wave height in front and behind the column are measured.
        \begin{figure}[h!]
            \centering
            \includegraphics[width=0.5\textwidth]{./figures/Diagram_case}
            \caption{Diagram of experiment apparatus}
        \end{figure}


\section{Implementation steps}
In a first step we plan to study some relavant chapters in textbook, including multidimensional hyperbolic problems, multidimensional numerical methods and implementation of multidimensional system in CLAWPACK, especially the shallow water equations. 
\par
Next we plan to model a dambreak problem in CLAWPACK, using the shallow water equation.
\par
Then we will investigate on how to put a square column into the mesh in CLAWPACK. Maybe cutting out a section of the grid where the column is and modify the code to make it treat the interface as boundary. 
\par
Finally all simulation result will be compared to experiment result.
\section{Validation Procedure}
The validation procedure will include comparison of wave height and streamwise velocity, U, at different depth, at where the center of column should be located in case without a column, and comparison of wave height in front of the column.
\end{itemize}
{\footnotesize
\begin{thebibliography}{100}
\bibitem{HA} Halld\'or \'Arnason  (2005). Thesis - Interactions between an incident bore and a free-standing coastal structure. \url{http://faculty.washington.edu/cpetroff/Halldor%20dissertation.pdf}
\end{thebibliography}
}
\end{document}

