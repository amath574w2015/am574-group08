\documentclass[11pt]{article}

\usepackage{graphicx}
\usepackage{amsmath,amsfonts,amssymb}

\usepackage{hyperref}  % for urls and hyperlinks


\setlength{\textwidth}{6.2in}
\setlength{\oddsidemargin}{0.3in}
\setlength{\evensidemargin}{0in}
\setlength{\textheight}{8.7in}
\setlength{\voffset}{-.7in}
\setlength{\headsep}{26pt}
\setlength{\parindent}{0pt}
\setlength{\parskip}{5pt}

% input some useful macros from RJLmacros.tex:
\input{../macros/RJLmacros}

\title{Comparison of the solution of the shallow equation with experimental}

\author{Xinsheng Qin xsqin@uw.edu\\
        Kaspar Mueller kasparm@uw.edu}

\begin{document}

\maketitle

\begin{abstract}
In this paper we compare the solution of the shallow water equations with experimental results. The experiment simulates the interaction between an incident bore and a free-standing coastal structure. The shallow water equations are solved using the CLAWPACK software.
\end{abstract}

\section{Introduction}\label{Sec:intro}
The aim of this work is to compare computational results using CLAWPACK with experimental results. This is done for the interaction of a bore and a free-standing coastal structure. The experiment results are given by Halld\'or \'Arnason \cite{HA}. We model the water using the shallow water equations which will be introduced in Section \ref{Sec:ShallowWater}. We solve these equations using CLAWPACK's GeoClaw solver which is specialized for geophysical flow problems. In Section \ref{Sec:Algo} we show how GeoClaw solves the shallow water equations and how it handles the topography. In Section \ref{Sec:Setup} we describe the setup of the test case. We focus on three cases. First we compare wave height and velocity for the pure dambreak without structure in the domain in Section \ref{Sec:Zero}. Second we add a square column and compare wave height upstream as well as downstream of the column in \ref{Sec:Square}. Lastly in Section \ref{Sec:Cylinder} we compare the wave height at the same location but for a cylinder column. We close this paper with a conclusion in Section \ref{Sec:Conclusion}.
\section{The shallow water equations}\label{Sec:ShallowWater}
We write the 2D depth averaged shallow water equations as
\eqm
h_t + (uh)_x + (vh)_y = 0\\
(hu)_t + (huv)_y +(hu^2 + \frac{1}{2}gh^2)_x = -ghB_x - Du\\
(hv)_t + (huv)_x +(hv^2 + \frac{1}{2}gh^2)_y = -ghB_y - Dv
\enm
where $u,v$ are the horizontal depth averaged velocities, $B$ the topography and $D$ the drag coefficient. $g$ denotes the gravitational acceleration. In the present case the topography will be flat except the column located downstream. The drag coefficient $D$ is given by
\eq
D = \frac{gM^2\sqrt{(u^2+v^2)}}{h^{5/3}}
\en
where $M$ is the Manning coefficient which is $M = 0.015$. The Manning coefficient is an empirical coefficient that includes the roughness of the surface. The initial condition is given by a jump in the water depth $h$. The water is initially at rest.
\section{The Algorithm in CLAWPACK/GeoClaw}\label{Sec:Algo}
We solve the shallow water equations with CLAWPACK. Specifically we used the GeoClaw solver which used a special version of the f-wave approach described in \cite{DLG}. This solver combines the qualities of the Roe solver, HLLE-type solver and the f-wave approach. From the Roe solver it has the quality of giving a exact solution for a single-shock Riemann problem. Like the HLLE solver this method is depth positive semidefinite. The method has also a natural entropy-fix and gives a better approximation for problems with large rarefactions. The solver also maintains large class of steady states.
The topography and drag term are incorporated in the Riemann problem. Stability and accuracy are improved by solving a transverse Riemann problem, where the waves moving normal to the cell are split into traverse direction.\\
In the GeoClaw solver adaptive mesh refinement (AMR) can be used. Multiple levels can be use where for each level the domain is specified. The topography for the problem is defined in a file. The topography goes into the shallow water equations through the right hand side $B$. The value from the file are converted into cell averages using a picewise bilinear function that interpolates the pointwise values. $B$ is then obtained by evaluating the exact integral of this function over the cell. 
\section{The setup of the test case}\label{Sec:Setup}
The benchmark experiment was conducted in a 16.6 m long, 0.6m wide and 0.45 m deep wave tank. When a gate is lifted, it generates a single bore to impinge the column located downstream.
We are going to simulate 2 cases in the experiment. The 1st case purely generated the single bore, the 2nd case generated the bore to impinge the column with one side facing the flow.
\par
In the 1st case, wave height and streamwise velocity, U, at different depth, at where the center of column should be located are measured. In the 2nd case, the wave height in front and behind the column are measured.
        \begin{figure}[h!]
            \centering
            \includegraphics[width=0.5\textwidth]{../proposal/figures/Diagram_case}
            \caption{Diagram of experiment apparatus}
        \end{figure}

In all cases below, boundary conditions are all set to reflection wall. 
\section{Dam Break}\label{Sec:Zero}
The initial condition for this and the following cases is given by
\eqm
h_0(x,y)=& 
\begin{cases}
0.25 & x<5.9 \\
0.02 & \text{else}
\end{cases}\\
u_0(x,y) =& 0
\enm

Figure \ref{fig:waveheight_nocolumn} shows the time history of the wave height measured at the center location where the column will be mounted. 
Result from geoclaw agree quite well in general. At around t=3.8, geoclaw accurately predicted the arrival of the shock. 
As for the peak value of this shock, geoclaw overestimated it a little, while OpenFOAM, another open source computational fluid dynamics software gave a better value by solving the Navier-Stokes equations with turbulent model.
After a longer time, the wave gets reflected at the right wall of the tank, which can be seen by the second jump in wave height in the figure. 
Geoclaw also predicted the arrival time of this peak quite well and got only a little overestimation of the peak value.
\begin{figure}[h!]
    \centering
    \includegraphics[width=0.5\textwidth]{./plots/waveheight_nocolumn}
    \caption{time history of wave height for case without a column}
    \label{fig:waveheight_nocolumn}
\end{figure}

Figure \ref{fig:velocityU_nocolumn} gives the time history of the streamwise velocity $u$ from geoclaw and experiment. 
Note that this velocity $u$ in both geoclaw and experiment are depth-averaged. 
Geoclaw accurately predicted the time of the velocity peak as well as the value of the peak. 
After the peak arrived, geoclaw overestimated the value of velocity.
The reason why geoclaw gives a higher velocity may be attributed to its depth-averaged property: water at all depth are propagating with the same speed. In the experiment, the velocity gets smaller as it comes closer to the bottom. 
\begin{figure}[h!]
    \centering
    \includegraphics[width=0.5\textwidth]{./plots/velocityU_nocolumn}
    \caption{time history of streamwise velocity U for case without a column}
    \label{fig:velocityU_nocolumn}
\end{figure}




\section{Dam Break with square column}\label{Sec:Square}
Figure \ref{fig:waveheight_square_x=11.0} gives a comparison of the time history of the wave height measured at 0.04m ahead of the leading edge of the square column.(0.1m ahead of center of the column)
Note that this figure is from t=3.0 to t=4.0, which is quite a short range compared to the previous figures.
There are two jumps in result given by geoclaw. First jump change water level from around 0.02 to 0.1, second one jump from 0.1 to a higher level of around 0.22.
If we recall that the wave height is around 0.1 in \ref{fig:waveheight_nocolumn}, we can confirm this first jump is cause by the arrival of shock wave caused by dambreak. The second jump is induced by wave reflected by the column.  
If we compare the time interval between first jump and second jump with that in \ref{fig:waveheight_square_x=11.02}, we can see that as the gauge gets closer to column, this time interval also gets shorter.
The arrival time of the peak predicted by geoclaw is about 0.2 second earlier than the time measured from experiment and in experiment, the jump is more like a a smooth increase. 
This can be explained by the difference between the shallow water equations that geoclaw solves and N-S equations that better describe the physics.
In shallow water equations, fluid viscosity is not taken into account, while it's significant in N-S equation (or real physics).
This causes the shock remain its shape in as it propagates in shallow water equation and smooth out in experiment.
Thus we see a jump in result from shallow water equation while a smooth increase in experiment result.
Shallow water equation is essentially a hyperbolic PDE, whose solution depends on initial condition along characteristics in spatio-temporal domain. 
Thus before the propagating shock arrive, all water downstream the shock stays still. In other words, it even does not know there is a column before the shock hits the column. In contrast, N-S equation is a elliptic PDE. 
In real physics govened by N-S equation, when front of the shock wave gets close to the column, water downstream it and upstream column are affected and resist the comming of wave to some extent, which raises the wave hegiht even before the wave hits the column.   
This explains why we cannot see two jumps in experiment results.
\begin{figure}[h!]
    \centering
    \includegraphics[width=0.5\textwidth]{./plots/waveheight_square_x11}
    \caption{time history of wave height for case with square column}
    \label{fig:waveheight_square_x=11.0}
\end{figure}
\begin{figure}[h!]
    \centering
    \includegraphics[width=0.5\textwidth]{./plots/waveheight_square_x1102}
    \caption{time history of wave height for case with square column, at x = 11.02}
    \label{fig:waveheight_square_x=11.02}
\end{figure}



\section{Dam Break with cylinder column}\label{Sec:Cylinder}
Figure \ref{fig:waveheight_cylinder_x=11.0} is the time history of the wave height measured at 0.04m ahead of the leading edge of cylinder column.(0.1m ahead of the center of the column)
Since the way geoclaw treat a cell partly belong to cylinder and partly belong to tank bottom is to give that cell an average value of topography from cylinder and tank bottom based on area ratio, we get a quasi-circle actually instead of circle. Thus the result in this case does not agree well with the experiment result. 
Future test can be conducted by setting a higher level for mesh refinement to get a better circle.
\begin{figure}[h!]
    \centering
    \includegraphics[width=0.5\textwidth]{./plots/waveheight_cylinder_x11}
    \caption{time history of wave height for case with cylinder column}
    \label{fig:waveheight_cylinder_x=11.0}
\end{figure}

\section{Conclusion}\label{Sec:Conclusion}
{\footnotesize
\begin{thebibliography}{100}
\bibitem{HA} Halld\'or \'Arnason  (2005). Thesis - Interactions between an incident bore and a free-standing coastal structure. \url{http://faculty.washington.edu/cpetroff/Halldor%20dissertation.pdf}
\bibitem{DLG} David L. George (2008) Journal of Computational Physics - Augmented Riemann solvers for the shallow water equations over variable topography with steady states and inundation.
\bibitem{MJB} Marsha J. Berger and David L. George and Randall J. LeVeque and Kyle T. Mandli (2011) Advances in Water Resources - The GeoClaw software for depth-averaged flows with adaptive refinement.
\end{thebibliography}
}
\end{document}

